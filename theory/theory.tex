\documentclass{article}

\newcommand{\commandprependpath}{style/}
\usepackage{style/style}

\begin{document}

We implement the method of harmonic balance using exponential periodic basis functions instead of trigonometric.

Assume the state $\x(t): \R \rightarrow \R^n$ is governed by
\begin{align}
  M \ddot \x + C \dot \x + K \x + \f\ts{nl}(\x, \dot \x) = \f\ts{ext}(\omega, t).
\end{align}
Note that $n$ is the number of degrees of freedom;
$M$, $C$, and $K$ are the mass, damping, and stiffness matrices, respectively;
$\f\ts{nl}$ is the internal, potentially nonlinear force;
$\f\ts{ext}$ is the external force;
and $\omega$ is the fundamental frequency of the external force.

Then $x$ is periodic with period $2 \pi / \omega$, so we can estimate it as a finite sum of exponential periodic basis functions:
\begin{align}
  \x(t) \approx \sum_{k=-N_H}^{N_H} \ab_k \exp\paren{i k \omega t}.
\end{align}
We can write the forces similarly:
\begin{align}
  \f\ts{nl}(\x, \dot \x)
  &\approx \sum_{k=-N_H}^{N_H} \cb_k \exp\paren{i k \omega t} \\
  \f\ts{ext}(t)
  &\approx \sum_{k=-N_H}^{N_H} \db_k \exp\paren{i k \omega t}.
\end{align}
Collect the Fourier coefficients:
\begin{align}
  \z &\coloneq
  \begin{bmatrix}
    \ab_{-N_H}\trp & \cdots & \ab_{-1}\trp &
    \ab_0\trp & \ab_1\trp & \cdots & \ab_{N_H}\trp
  \end{bmatrix}\trp \in \C^{\paren{2 N_H + 1} n \times 1} \\
  \bb\ts{nl}(\z) &\coloneq
  \begin{bmatrix}
    \cb_{-N_H}\trp & \cdots & \cb_{-1}\trp &
    \cb_0\trp & \cb_1\trp & \cdots & \cb_{N_H}\trp
  \end{bmatrix}\trp \\
  \bb\ts{ext} &\coloneq
  \begin{bmatrix}
    \db_{-N_H}\trp & \cdots & \db_{-1}\trp &
    \db_0\trp & \db_1\trp & \cdots & \db_{N_H}\trp
  \end{bmatrix}\trp.
\end{align}
Also let $\bb(\z) = \bb\ts{ext} - \bb\ts{nl}(\z)$ denote the Fourier coefficients of the force $\f(t, \x, \dot \x) = \f\ts{ext}(t) - \f\ts{nl}(\x, \dot \x)$.

Collect the exponential periodic basis functions
\begin{align}
  Q(t) \coloneq
  \begin{bmatrix}
    e^{-i N_H \omega t} & \cdots & e^{-i \omega t} & 1 &
    e^{i \omega t} & \cdots & e^{i N_H \omega t}
  \end{bmatrix} \in \C^{1 \times (2 N_H + 1)}
\end{align}
so that
\begin{align}
  \x(t)
  &= (Q(t) \otimes I_n) \z \\
  \f\ts{nl}(\x, \dot \x)
  &= (Q(t) \otimes I_n) \bb\ts{nl} \\
  \f\ts{ext}(t)
  &= (Q(t) \otimes I_n) \bb\ts{ext} \\
  \f(t, \x, \dot \x)
  &= (Q(t) \otimes I_n) \bb(\z)
\end{align}
where $\otimes$ denotes the Kronecker product.
Expanding the Kronecker product to make sure we're not pulling our own leg,
\begin{gather*}
  (Q(t) \otimes I_n) \\
  =
  \begin{bmatrix}
    e^{-i N_H \omega t} I_n & \cdots & e^{-i \omega t} I_n & I_n &
    e^{i \omega t} I_n & e^{i N_H \omega t} I_n
  \end{bmatrix} \in \C^{\paren{2 N_H + 1} n \times n}
\end{gather*}
so
\begin{gather*}
  (Q(t) \otimes I_n) \z \\
  =
  \begin{bmatrix}
    e^{-i N_H \omega t} I_n & \cdots & e^{-i \omega t} I_n & I_n &
    e^{i \omega t} I_n & e^{i N_H \omega t} I_n
  \end{bmatrix}
  \begin{bmatrix}
    \ab_{-N_H} \\ \vdots \\ \ab_{-1} \\
    \ab_0 \\ \ab_1 \\ \vdots \\ \ab_{N_H}
  \end{bmatrix} \\
  = \x(t).
\end{gather*}

\section{Differentiating}

Suppose we want to write the Fourier series for $\dot \x$?
Observe,
\begin{align*}
  \dot \x(t)
  &= \frac d{dt} \x
  = \frac d{dt} (Q(t) \otimes I_n) \z \\
  &= \paren*{\dot Q(t) \otimes I_n} \z.
\end{align*}
Letting
\begin{align*}
  \widetilde \nabla
  &\coloneq
  \begin{bmatrix}
    -i N_H & \cdots -i & 0 & i & \cdots & i N_H
  \end{bmatrix} \\
  \nabla &\coloneq \diag{\widetilde \nabla}
\end{align*}
we find that
\begin{align*}
  \dot Q(t)
  &=
  \omega
  \begin{bmatrix}
    -i N_H e^{-i N_H \omega t} & \cdots & -i e^{-i \omega t} & 0 &
    i e^{i \omega t} & \cdots & i N_H e^{i N_H \omega t}
  \end{bmatrix} \\
  &= \omega Q(t) \odot \widetilde \nabla \\
  &= \omega Q(t) \nabla.
\end{align*}
Likewise,
\begin{align*}
  \ddot Q(t)
  &= \omega^2 Q(t) \nabla^2.
\end{align*}
This yields the identities
\begin{align}
  \x(t) &= \paren{Q(t) \otimes I_n} \z \\
  \dot \x(t) &= \omega \paren{Q(t) \nabla \otimes I_n} \z \\
  \ddot \x(t) &= \omega^2 \paren*{Q(t) \nabla^2 \otimes I_n} \z.
\end{align}

\section{It's gettin' real}

Since $\x$ is real, we know $\ab_{-k} = \conj{\ab_k}$ and $\exp\paren{-i k \omega t} = \conj{\exp\paren{i k \omega t}}$, so we can write
\begin{align*}
  \x(t)
  &\approx \sum_{k=-N_H}^{N_H} \ab_k \exp\paren{i k \omega t} \\
  &= \ab_0 + \sum_{k=0}^{N_H}
  \ab_k \exp\paren{i k \omega t}
  + \ab_{-k} \exp\paren{-i k \omega t} \\
  &= \ab_0 + \sum_{k=0}^{N_H}
  \ab_k \exp\paren{i k \omega t}
  + \conj{\ab_k \exp\paren{i k \omega t}} \\
  &= \ab_0 + 2 \sum_{k=0}^{N_H} \Re\bbrack{\ab_k \exp\paren{i k \omega t}}.
\end{align*}
We can write the forces similarly:
\begin{align*}
  \f\ts{nl}(\x, \dot \x)
  &= \cb_0 + 2 \sum_{k=0}^{N_H} \Re\bbrack{\cb_k \exp\paren{i k \omega t}}\\
  \f\ts{ext}(t)
  &= \db_0 + 2 \sum_{k=0}^{N_H} \Re\bbrack{\db_k \exp\paren{i k \omega t}}.
\end{align*}
Collect the Fourier coefficients:
\begin{align*}
  \widehat \z &\coloneq
  \begin{bmatrix}
    \ab_0\trp & \ab_1\trp & \cdots & \ab_{N_H}\trp
  \end{bmatrix}\trp \in \C^{n\paren{N_H + 1}} \\
  \widehat \bb\ts{nl} &\coloneq
  \begin{bmatrix}
    \cb_0\trp & \cb_1\trp & \cdots & \cb_{N_H}\trp
  \end{bmatrix}\trp \\
  \widehat \bb\ts{ext} &\coloneq
  \begin{bmatrix}
    \db_0\trp & \db_1\trp & \cdots & \db_{N_H}\trp
  \end{bmatrix}\trp.
\end{align*}

\end{document}