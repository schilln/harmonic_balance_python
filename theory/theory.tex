\documentclass{article}

\newcommand{\commandprependpath}{style/}
\usepackage{style/style}

\usepackage[margin=1in]{geometry}
\usepackage{hyperref}

\newcommand{\wh}{\widehat}

\begin{document}

\section{Introduction}

We implement the method of harmonic balance using exponential periodic basis functions instead of trigonometric.

Assume the state $\x(t): \R \rightarrow \R^n$ is governed by
\begin{align}
  M \ddot \x + C \dot \x + K \x + \f\ts{nl}(\x, \dot \x) = \f\ts{ext}(t, \omega).
  \label{eq:system-time}
\end{align}
Note that $n$ is the number of degrees of freedom;
$M$, $C$, and $K$ are the mass, damping, and stiffness matrices, respectively;
$\f\ts{nl}$ is the internal, potentially nonlinear force;
$\f\ts{ext}$ is the external force;
and $\omega$ is the fundamental frequency of the external force.

Then $\x$ is periodic with period $2 \pi / \omega$, so we can estimate it as a finite sum of exponential periodic basis functions:
\begin{align}
  \x(t, \omega) \approx \sum_{k=-N_H}^{N_H} \ab_k \exp\paren{i k \omega t}.
\end{align}
We can write the forces similarly:
\begin{align}
  \f\ts{nl}(\x, \dot \x, \omega)
  &\approx \sum_{k=-N_H}^{N_H} \cb_k \exp\paren{i k \omega t} \\
  \f\ts{ext}(t, \omega)
  &\approx \sum_{k=-N_H}^{N_H} \db_k \exp\paren{i k \omega t}.
\end{align}

\section{Projecting the governing equation}

Collect the Fourier coefficients:
\begin{align}
  \z(\omega) &\coloneq
  \begin{bmatrix}
    \ab_{-N_H}\trp & \cdots & \ab_{-1}\trp &
    \ab_0\trp & \ab_1\trp & \cdots & \ab_{N_H}\trp
  \end{bmatrix}\trp \in \C^{\paren{2 N_H + 1} n \times 1} \\
  \bb\ts{nl}(\z) &\coloneq
  \begin{bmatrix}
    \cb_{-N_H}\trp & \cdots & \cb_{-1}\trp &
    \cb_0\trp & \cb_1\trp & \cdots & \cb_{N_H}\trp
  \end{bmatrix}\trp \\
  \bb\ts{ext}(\omega) &\coloneq
  \begin{bmatrix}
    \db_{-N_H}\trp & \cdots & \db_{-1}\trp &
    \db_0\trp & \db_1\trp & \cdots & \db_{N_H}\trp
  \end{bmatrix}\trp.
\end{align}
Also let $\bb(\z, \omega) = \bb\ts{ext}(\omega) - \bb\ts{nl}(\z)$ denote the Fourier coefficients of the force $\f(t, \x, \dot \x, \omega) = \f\ts{ext}(t, \omega) - \f\ts{nl}(\x, \dot \x, \omega)$.

Collect the exponential periodic basis functions
\begin{align}
  Q(t, \omega) \coloneq
  \begin{bmatrix}
    e^{-i N_H \omega t} & \cdots & e^{-i \omega t} & 1 &
    e^{i \omega t} & \cdots & e^{i N_H \omega t}
  \end{bmatrix} \in \C^{1 \times (2 N_H + 1)}
\end{align}
so that
\begin{align}
  \x(t, \omega)
  &= (Q(t, \omega) \otimes I_n) \z(\omega) \\
  \f\ts{nl}(\x, \dot \x, \omega)
  &= (Q(t, \omega) \otimes I_n) \bb\ts{nl}(\z) \\
  \f\ts{ext}(t, \omega)
  &= (Q(t, \omega) \otimes I_n) \bb\ts{ext}(\omega) \\
  \f(t, \x, \dot \x, \omega)
  &= (Q(t, \omega) \otimes I_n) \bb(\z, \omega)
\end{align}
where $\otimes$ denotes the Kronecker product.
Expanding the Kronecker product to make sure we're not pulling our own leg,
\begin{gather*}
  (Q(t, \omega) \otimes I_n) \\
  =
  \begin{bmatrix}
    e^{-i N_H \omega t} I_n & \cdots & e^{-i \omega t} I_n & I_n &
    e^{i \omega t} I_n & e^{i N_H \omega t} I_n
  \end{bmatrix} \in \C^{\paren{2 N_H + 1} n \times n}
\end{gather*}
so
\begin{gather*}
  (Q(t, \omega) \otimes I_n) \z(\omega) \\
  =
  \begin{bmatrix}
    e^{-i N_H \omega t} I_n & \cdots & e^{-i \omega t} I_n & I_n &
    e^{i \omega t} I_n & e^{i N_H \omega t} I_n
  \end{bmatrix}
  \begin{bmatrix}
    \ab_{-N_H} \\ \vdots \\ \ab_{-1} \\
    \ab_0 \\ \ab_1 \\ \vdots \\ \ab_{N_H}
  \end{bmatrix} \\
  = \x(t, \omega).
\end{gather*}

\subsection{Differentiating}

Suppose we want to write the Fourier series for $\dot \x$?
Observe,
\begin{align*}
  \dot \x(t, \omega)
  &= \frac d{dt} \x(t, \omega)
  = \frac d{dt} (Q(t, \omega) \otimes I_n) \z(\omega) \\
  &= \paren*{\dot Q(t, \omega) \otimes I_n} \z(\omega).
\end{align*}
Letting
\begin{align}
  \widetilde \nabla
  &\coloneq
  \begin{bmatrix}
    -i N_H & \cdots -i & 0 & i & \cdots & i N_H
  \end{bmatrix} \\
  \label{eq:nabla}
  \nabla &\coloneq \diag{\widetilde \nabla}
\end{align}
we find that
\begin{align*}
  \dot Q(t, \omega)
  &=
  \omega
  \begin{bmatrix}
    -i N_H e^{-i N_H \omega t} & \cdots & -i e^{-i \omega t} & 0 &
    i e^{i \omega t} & \cdots & i N_H e^{i N_H \omega t}
  \end{bmatrix} \\
  &= \omega Q(t, \omega) \odot \widetilde \nabla \\
  &= \omega Q(t, \omega) \nabla.
\end{align*}
Likewise,
\begin{align*}
  \ddot Q(t, \omega)
  &= \omega^2 Q(t, \omega) \nabla^2.
\end{align*}
This yields the identities
\begin{align}
  \label{eq:x(z)}
  \x(t, \omega) &= \paren{Q(t, \omega) \otimes I_n} \z(\omega) \\
  \label{eq:xp(z)}
  \dot \x(t, \omega) &= \omega \paren{Q(t, \omega) \nabla \otimes I_n} \z(\omega) \\
  \label{eq:xpp(z)}
  \ddot \x(t, \omega) &= \omega^2 \paren*{Q(t, \omega) \nabla^2 \otimes I_n} \z(\omega).
\end{align}

\subsection{Performing the projection}
\label{sec:projection}

First, we rewrite \eqref{eq:system-time} using \eqref{eq:x(z)}, \eqref{eq:xp(z)}, and \eqref{eq:xpp(z)}, beginning with the first and using the mixed-product property of the Kronecker product:
\begin{align}
  \nonumber
  &\phantom{{}={}} M \ddot \x \\
  \nonumber
  &= M \omega^2 \paren*{Q \nabla^2 \otimes I_n} \z \\
  \nonumber
  &= \omega^2 (1 \otimes M) \paren*{Q \nabla^2 \otimes I_n} \z \\
  \nonumber
  &= \omega^2 \bbrack*{\paren*{1 Q \nabla^2} \otimes (M I_n)} \z \\
  &= \omega^2 \bbrack*{\paren*{Q \nabla^2} \otimes M} \z.
\end{align}
The same computations show
\begin{align}
  C \dot \x &= \omega \bbrack*{\paren*{Q \nabla} \otimes C} \z \\
  K \x &= \bbrack*{Q \otimes K} \z.
\end{align}
Now the system equation \eqref{eq:system-time} becomes (suppressing $Q$'s dependence on $\omega$)
\begin{align}
  \label{eq:system-time-kron}
  \bbrack*{
    \omega^2 \paren*{Q(t) \nabla^2} \otimes M
    + \omega \paren*{Q(t) \nabla} \otimes C
    + Q(t) \otimes K
  }\z(\omega)
  = (Q(t) \otimes I_n) \bb(\z).
\end{align}

Next, we project the system equation from the time domain \eqref{eq:system-time} to the frequency domain via Fourier--Galerkin projection (I think).
Define the inner product on functions $f: [0, 2 \pi / \omega] \rightarrow \C$:
\begin{align}
  \inpr{f, g} \coloneq \frac \omega{2 \pi} \int_0^{2 \pi / \omega} \conj{f(t)} g(t) dt.
\end{align}
Now let $-N_H \le k, \ell \le N_H$, and write $\hat k \coloneq k + N_H$ and $\hat \ell \coloneq \ell + N_H$.
Then
\begin{align*}
  &\phantom{{}={}} \frac \omega{2 \pi} \paren*{\int_0^{2 \pi / \omega} Q(t)^* Q(t) dt}_{\hat k \hat \ell} \\
  &= \frac \omega{2 \pi} \int_0^{2 \pi / \omega} \paren*{Q(t)^* Q(t)}_{\hat k \hat \ell} dt \\
  &= \frac \omega{2 \pi} \int_0^{2 \pi / \omega} \exp\paren{-i k \omega t} \exp\paren{i \ell \omega t} dt \\
  &= \inpr{\exp\paren{i k \omega t},\; \exp\paren{i \ell \omega t}} \\
  &= \delta_{k \ell}
\end{align*}
since the basis functions are orthonormal in this inner product.
Then
\begin{align*}
  \frac \omega{2 \pi} \paren*{\int_0^{2 \pi / \omega} Q(t)^* Q(t) dt}
  = I_{2 N_H + 1}.
\end{align*}
Now we integrate \eqref{eq:system-time-kron} against $Q(t)^*$ to obtain an algebraic system of equations in the frequency domain:
\begin{align*}
  &\phantom{{}={}} \frac \omega{2 \pi} \int_0^{2 \pi / \omega} Q(t)^* \big[
    \omega^2 \paren*{Q(t) \nabla^2} \otimes M
  \big] \, dt \\
  &= \omega^2 \frac \omega{2 \pi} \int_0^{2 \pi / \omega} \big[
    Q(t)^* \otimes 1
  \big] \big[
    \paren*{Q(t) \nabla^2} \otimes M
  \big] \, dt \\
  &= \omega^2 \frac \omega{2 \pi} \int_0^{2 \pi / \omega} \big[
    Q(t)^* \paren*{Q(t) \nabla^2}
  \big] \otimes \big[
    1 M
  \big] \, dt \\
  &= \omega^2 \bbrack*{
    \frac \omega{2 \pi}
    \int_0^{2 \pi / \omega}
    Q(t)^* Q(t) \, dt \; \nabla^2
  } \otimes M \\
  &=  \omega^2 \big[ I_{2 N_H + 1} \; \nabla^2 \big] \otimes M \\
  &= \omega^2 \nabla^2 \otimes M.
\end{align*}
Following the same process for the other terms in \eqref{eq:system-time-kron}, we at last obtain
\begin{align}
  \label{eq:system}
  \bbrack*{
    \omega^2 \nabla^2 \otimes M
    + \omega \nabla \otimes C
    + I_{2 N_H + 1} \otimes K
  } \z
  = \bb(\z)
\end{align}
and define $A(\omega) \coloneq \bbrack*{
  \omega^2 \nabla^2 \otimes M
  + \omega \nabla \otimes C
  + I_{2 N_H + 1} \otimes K
}$.

\section{Computing the nonlinear force}

How do we compute the Fourier coefficients of the nonlinear force, $\bb(\z)$?
One method is the alternating frequency/time method (AFT), in which we take the frequency domain solution $\z$, send it to the time domain to obtain $\x$ and $\dot \x$, pass these through the nonlinear force to obtain $\f\ts{nl}(\x, \dot \x)$, and finally send this back to the frequency domain to obtain $\bb\ts{nl}$.

We also use a fancy technique called trigonometric collocation in which we construct discrete Fourier transform (DFT) matrices using extra time samples DFT so that we can capture information from higher frequencies resulting from the nonlinearity.
(For example, consider passing $e^{i \omega t}$ through a cubic nonlinearity $y \mapsto y^3$, yielding $e^{i 3 \omega t}$.)

\subsection{Computing time from frequency}

Assume we use $N$ samples evenly spaced in one period, $t_j = \frac{2 \pi}\omega \frac jN$, $j = 0, 1, \ldots, N - 1$.
Define
\begin{align}
  \q(k)
  &\coloneq
  \begin{bmatrix}
    e^{i k \omega t_0} \\
    e^{i k \omega t_1} \\
    \vdots \\
    e^{i k \omega t_{N-1}}
  \end{bmatrix} \\
  \Gamma
  &\coloneq
  \begin{bmatrix}
    I_n \otimes \q(-N_H) & \cdots & I_n \otimes \q(0) & I_n \otimes \q(1) & \cdots & I_n \q(N_H)
  \end{bmatrix}
\end{align}
Now
\begin{align*}
  \Gamma \z
  &= (I_n \otimes \q(-N_H)) \ab_{-N_H} + \cdots + (I_n \otimes \q(N_H)) \ab_{N_H}.
\end{align*}
Examining a single summand, we find
\begin{align*}
  [I_n \otimes \q(k)] \ab_k
  &=
  \begin{bmatrix}
    \q(k) & & \\
    & \ddots & \\
    & & \q(k)
  \end{bmatrix}
  \begin{bmatrix}
    a_{k, 0} \\
    \vdots \\
    a_{k, n-1}
  \end{bmatrix} \\
  &=
  \begin{bmatrix}
    \q(k) a_{k, 0} \\
    \vdots \\
    \q(k) a_{k, n-1}
  \end{bmatrix}
\end{align*}
where $a_{k, d}$ is the $k$th Fourier coefficient of the $d$th degree of freedom of $\x$.
Then
\begin{align*}
  \Gamma \z =
  \begin{bmatrix}
    \sum_{k=-N_H}^{N_H} \q(k) a_{k, 0} \\
    \vdots \\
    \sum_{k=-N_H}^{N_H} \q(k) a_{k, n-1}
  \end{bmatrix}
  =
  \begin{bmatrix}
    \x_0 \\
    \vdots \\
    \x_{n-1}
  \end{bmatrix}
\end{align*}
where $\x_d =
\begin{bmatrix}
  x_d(t_0) & x_d(t_1) & \cdots & x_d(t_{N-1})
\end{bmatrix}
$
contains the values of the $d$th degree of freedom of $\x(t)$ at $t = t_0, \ldots, t_{N-1}$.
Write
\begin{align*}
  \x_s \coloneq \Gamma \z
\end{align*}
to denote this sampling of $\x$ at $N$ points in time.

\subsection{Computing frequency from time}

We now guess at the form of $\Gamma^\dagger$, the pseudo-inverse of $\Gamma$, and check that it in fact computes $\z = \Gamma^\dagger \x_s$
We try
\begin{align}
  \Gamma^\dagger \coloneq \frac1N
  \begin{bmatrix}
    I_n \otimes \q(-N_H)^* \\
    \vdots \\
    I_n \otimes \q(N_H)^*
  \end{bmatrix}.
\end{align}
First note
\begin{align*}
  \frac1N (I_n \otimes \q(k)^*) \x_s
  &= \frac1N
  \begin{bmatrix}
    \q(k)^* \\
    & \q(k)^* \\
    & & \ddots \\
    & & & \q(k)^*
  \end{bmatrix}
  \begin{bmatrix}
    \x_0 \\
    \x_1 \\
    \vdots \\
    \x_{n-1}
  \end{bmatrix} \\
  &= \frac1N
  \begin{bmatrix}
    \q(k)^* \x_0 \\
    \q(k)^* \x_1 \\
    \ddots \\
    \q(k)^* \x_{n-1}.
  \end{bmatrix}
\end{align*}
Now observe that
\begin{align*}
  \frac1N \q(k)^* \x_d
  &= \frac1N
  \begin{bmatrix}
    e^{-i k \omega t_0} & e^{-i k \omega t_1} & \cdots & e^{-i k \omega t_{N-1}}
  \end{bmatrix}
  \begin{bmatrix}
    x_d(t_0) \\
    x_d(t_1) \\
    \vdots \\
    x_d(t_{N-1})
  \end{bmatrix} \\
  &= \frac1N \sum_{j=0}^{N-1} e^{-i k \omega t_j} x_d(t_j) \\
  &\approx \frac \omega{2 \pi} \int_0^{2 \pi / \omega} e^{-i k \omega t} x_d(t) \, dt \\
  &= a_{k, d},
\end{align*}
the $k$th Fourier coefficient of the $d$th degree of freedom of $\x$.
Then
\begin{align*}
  \frac1N (I_n \otimes \q(k)^*) \x_s
  &=
  \begin{bmatrix}
    a_{k, 0} \\
    a_{k, 1} \\
    \vdots \\
    a_{k, n-1}
  \end{bmatrix}
  = \ab_k
\end{align*}
Now
\begin{align*}
  \Gamma^\dagger \x_s
  &= \frac1N
  \begin{bmatrix}
    I_n \otimes \q(-N_H)^* \\
    \vdots \\
    I_n \otimes \q(N_H)^*
  \end{bmatrix}
  \x_s \\
  &=
  \begin{bmatrix}
    I_n \otimes \q(-N_H)^* \x_s \\
    \vdots \\
    I_n \otimes \q(N_H)^* \x_s
  \end{bmatrix} \\
  &=
  \begin{bmatrix}
    \ab_{-N_H} \\
    \vdots \\
    \ab_{N_H}
  \end{bmatrix} \\
  &= \z,
\end{align*}
so we have $\z = \Gamma^\dagger \x_s$, as expected.

\subsection{Verifying the pseudo-inverse}

We now check that $\Gamma^\dagger \Gamma = I_{(2 N_H + 1) n}$ (assuming $N \ge 2 N_H + 1$).
Again using the mixed-product property of the Kronecker product, we note
\begin{align*}
  &\phantom{{}={}} \bbrack*{I_n \otimes \q(k)^*} \bbrack*{I_n \otimes \q(\ell)} \\
  &= I_n \otimes \bbrack*{\q(k)^* \q(\ell)}.
\end{align*}
Next observe that
\begin{align*}
  \frac1N \q(k)^* \q(\ell)
  &= \frac1N \sum_{j=0}^{N-1} e^{-i k \omega t_j} e^{i \ell \omega t_j} \\
  &\approx \frac \omega{2 \pi} \int_0^{2 \pi / \omega}
  e^{-i k \omega t} e^{i \ell \omega t} \, dt \\
  &= \inpr{e^{i k \omega t},\; e^{i \ell \omega t}} \\
  &= \delta_{k \ell}.
\end{align*}
Then
\begin{align*}
  &\phantom{{}={}} \Gamma^\dagger \Gamma \\
  &= \frac1N
  \begin{bmatrix}
    I_n \otimes \q(-N_H)^* \\
    \vdots \\
    I_n \otimes \q(N_H)^*
  \end{bmatrix}
  \begin{bmatrix}
    I_n \otimes \q(-N_H) & \cdots & I_n \q(N_H)
  \end{bmatrix} \\
  &= \frac1N
  \begin{bmatrix}
    I_n \otimes \bbrack*{\q(-N_H)^* \q(-N_H)} &
    I_n \otimes \bbrack*{\q(-N_H)^* \q(-N_H + 1)} &
    \cdots &
    I_n \otimes \bbrack*{\q(-N_H)^* \q(N_H)} \\
    I_n \otimes \bbrack*{\q(-N_H + 1)^* \q(-N_H)} &
    I_n \otimes \bbrack*{\q(-N_H + 1)^* \q(-N_H + 1)} &
    \cdots & \\
    \vdots & & \ddots & \\
    I_n \otimes \bbrack*{\q(N_H)^* \q(-N_H)} &
    \cdots &
    \cdots &
    I_n \otimes \bbrack*{\q(N_H)^* \q(N_H)} \\
  \end{bmatrix} \\
  &\approx
  \begin{bmatrix}
    I_n \otimes \delta_{-N_H, -N_H} & I_n \otimes \delta_{-N_H, -N_H + 1} & \cdots & I_n \otimes \delta_{-N_H, N_H} \\
    I_n \otimes \delta_{-N_H + 1, -N_H} & I_n \otimes \delta_{-N_H + 1, -N_H + 1} & \cdots & I_n \otimes \delta_{-N_H + 1, N_H} \\
    \vdots & & \ddots & \\
    I_n \otimes \delta_{N_H, -N_H} & I_n \otimes \delta_{N_H, -N_H + 1} & \cdots & I_n \otimes \delta_{N_H, N_H}
  \end{bmatrix} \\
  &= I_{(2 N_H + 1) n}.
\end{align*}

\subsection{Application}

Recall that given a function $y(t)$ with Fourier coefficients $\bbrace*{\wh y_k}_{k=-N_H}^{N_H}$, the Fourier coefficients of $\dot y(t)$ are $\bbrace*{i k \omega \wh y_k}_{k=-N_H}^{N_H}$.
If we collect the Fourier coefficients of $y$ into a vector $\wh \y =
\begin{bmatrix}
  \wh y_{-N_H} & \cdots & \wh y_{N_H}
\end{bmatrix}\trp,$
then using the definition of $\nabla$ \eqref{eq:nabla} we can write the Fourier coefficients of $\dot y$ as
\begin{align*}
  \wh {\dot \y}
  &=
  \begin{bmatrix}
    i (-N_H) \omega \wh y_{-N_H} & \cdots & i N_H \omega \wh y_{N_H}
  \end{bmatrix}\trp \\
  &=
  \omega \nabla
  \begin{bmatrix}
    \wh y_{-N_H} & \cdots & \wh y_{N_H}
  \end{bmatrix}\trp
\end{align*}

With the Fourier coefficients $\z =
\begin{bmatrix}
  \ab_{-N_H} & \cdots & \ab_{N_H}
\end{bmatrix}$
for the multiple-degree of freedom system $\x(t) \in \R^n$, we can write the Fourier coefficients for $\dot \x(t)$ as
\begin{align*}
  (\omega \nabla \otimes I_n) \z
  &= \omega
  \begin{bmatrix}
    i (-N_H) I_n & \\
    & i (-N_H + 1) I_n & \\
    & & \ddots & \\
    & & & i N_H I_n
  \end{bmatrix}
  \begin{bmatrix}
    \ab_{-N_H} \\
    \ab_{-N_H + 1} \\
    \vdots \\
    \ab_{N_H}
  \end{bmatrix} \\
  &= \omega
  \begin{bmatrix}
    i (-N_H) \ab_{-N_H} & \\
    & i (-N_H + 1) \ab_{-N_H + 1} & \\
    & & \ddots & \\
    & & & i N_H \ab_{N_H}
  \end{bmatrix}.
\end{align*}
Now we can write
\begin{align}
  \label{eq:x_s}
  \x_s &= \Gamma \z \\
  \label{eq:xp_s}
  \dot \x_s &= \omega \Gamma (\nabla \otimes I_n) \z \\
  \ddot \x_s &= \omega^2 \Gamma (\nabla^2 \otimes I_n) \z.
\end{align}
Compare with \eqref{eq:x(z)}, \eqref{eq:xp(z)}, \eqref{eq:xpp(z)}.

Importantly, we can now compute the Fourier coefficients of the nonlinear force $\f\ts{nl}(\x, \dot \x)$:
\begin{align}
  \label{eq:b_nl-aft}
  \bb\ts{nl}(\z)
  = \Gamma^\dagger \f\ts{nl} \big(
    \x_s,\; \dot \x_s
  \big)
  = \Gamma^\dagger \f\ts{nl} \big(
    \Gamma \z,\; \omega \Gamma (\nabla \otimes I_n) \z
  \big).
\end{align}

\section{Solving the equation}

Define the residual equation
\begin{align}
  \Rb(\z)
  &\coloneq \bbrack*{
    \omega^2 \nabla^2 \otimes M
    + \omega \nabla \otimes C
    + I_{2 N_H + 1} \otimes K
  } \z
  + \bb\ts{nl}(\z)
  - \bb\ts{ext} \\
  &= A(\omega) \z
  + \bb\ts{nl}(\z)
  - \bb\ts{ext}
\end{align}
measuring how well a solution $\z$ satisfies the dynamics.
If we have $\Rb_\z(\z) \coloneq d \Rb(\z) / d \z$, we can find roots of this equation using the Newton--Raphson method (and related algorithms).

We compute
\begin{align*}
  \Rb_\z(\z) = A(\omega) + \frac {d \bb\ts{nl}(\z)}{d \z}.
\end{align*}
Then using \eqref{eq:x_s}, \eqref{eq:xp_s}, and \eqref{eq:b_nl-aft},
\begin{align*}
  \frac {d \bb\ts{nl}(\z)}{d \z}
  &= \frac d{d \z} \bigg[
    \Gamma^\dagger \f\ts{nl} \paren{\x_s,\; \dot \x_s}
  \bigg] \\
  &= \Gamma^\dagger \frac d{d \z} \bigg[
    \f\ts{nl} \paren{\x_s,\; \dot \x_s}
  \bigg] \\
  &= \Gamma^\dagger \bigg[
    \frac d{d \x} \f\ts{nl} \paren{\x_s,\; \dot \x_s} \frac {d \x_s}{d \z}
    + \frac d{d \dot \x} \f\ts{nl} \paren{\x_s,\; \dot \x_s} \frac {d \dot \x_s}{d \z}
  \bigg] \\
  &= \Gamma^\dagger \bigg[
    \frac d{d \x} \f\ts{nl} \paren{\x_s,\; \dot \x_s} \Gamma
    + \omega \frac d{d \dot \x} \f\ts{nl} \paren{\x_s,\; \dot \x_s} \Gamma (\nabla \otimes I_n)
  \bigg].
\end{align*}

\section{It's getting real}

Since $\x$ is real, we know $\ab_{-k} = \conj{\ab_k}$ and $\exp\paren{-i k \omega t} = \conj{\exp\paren{i k \omega t}}$, so we can write
\begin{align*}
  \x(t)
  &\approx \sum_{k=-N_H}^{N_H} \ab_k \exp\paren{i k \omega t} \\
  &= \ab_0 + \sum_{k=1}^{N_H}
  \ab_k \exp\paren{i k \omega t}
  + \ab_{-k} \exp\paren{-i k \omega t} \\
  &= \ab_0 + \sum_{k=1}^{N_H}
  \ab_k \exp\paren{i k \omega t}
  + \conj{\ab_k \exp\paren{i k \omega t}} \\
  &= \ab_0 + 2 \sum_{k=1}^{N_H} \Re\bbrack{\ab_k \exp\paren{i k \omega t}}.
\end{align*}
Alternatively, if we define
\begin{align*}
  \wh \ab_k \coloneq
  \begin{cases}
    \ab_k / 2 & k = 0, \\
    \ab_k & k \ne 0,
  \end{cases}
\end{align*}
then since $\x$ is real-valued, we know that $\ab_0 \in \R$, and so we can write
\begin{align*}
  \x(t)
  &\approx \sum_{k=0}^{N_H} \wh \ab_k \exp\paren{i k \omega t} + \conj{\wh \ab_k \exp\paren{i k \omega t}} \\
  &= 2 \sum_{k=0}^{N_H} \Re \bbrack*{\wh \ab_k \exp\paren{i k \omega t}}
\end{align*}
We can write the forces similarly:
\begin{align*}
  \f\ts{nl}(\x, \dot \x)
  &= \cb_0 + 2 \sum_{k=1}^{N_H} \Re \bbrack{\cb_k \exp\paren{i k \omega t}} \\
  &= 2 \sum_{k=0}^{N_H} \Re \bbrack*{\wh \cb_k \exp\paren{i k \omega t}} \\
  \f\ts{ext}(t)
  &= \db_0 + 2 \sum_{k=1}^{N_H} \Re \bbrack{\db_k \exp\paren{i k \omega t}} \\
  &= 2 \sum_{k=0}^{N_H} \Re \bbrack*{\wh \db_k \exp\paren{i k \omega t}}.
\end{align*}
If we write $H \coloneq \diag(1/2, 1, \ldots, 1)_{N_H + 1} \otimes I_n$, then we can store the Fourier coefficients
\begin{align*}
  \wh \z &\coloneq
  \begin{bmatrix}
    \wh \ab_0\trp & \wh \ab_1\trp & \cdots & \wh \ab_{N_H}\trp
  \end{bmatrix}\trp \in \C^{n\paren{N_H + 1}} \\
  &=
  \begin{bmatrix}
    \ab_0\trp / 2 & \ab_1\trp & \cdots & \ab_{N_H}\trp
  \end{bmatrix}\trp = H \z \\
  \wh \bb\ts{nl} &\coloneq
  \begin{bmatrix}
    \wh \cb_0\trp & \wh \cb_1\trp & \cdots & \wh \cb_{N_H}\trp
  \end{bmatrix}\trp \\
  &=
  \begin{bmatrix}
    \cb_0\trp / 2 & \cb_1\trp & \cdots & \cb_{N_H}\trp
  \end{bmatrix}\trp = H \bb\ts{nl}
  \\
  \wh \bb\ts{ext} &\coloneq
  \begin{bmatrix}
    \wh \db_0\trp & \wh \db_1\trp & \cdots & \wh \db_{N_H}\trp
  \end{bmatrix}\trp \\
  &=
  \begin{bmatrix}
    \db_0\trp / 2 & \db_1\trp & \cdots & \db_{N_H}\trp
  \end{bmatrix}\trp = H \bb\ts{ext}
\end{align*}
and again write $\wh \bb(\z, \omega) = \wh \bb\ts{ext}(\omega) - \wh \bb\ts{nl}(\z)$.

\subsection{Projecting the governing equation}

Then writing
\begin{align}
  \wh Q(t, \omega) \coloneq
  \begin{bmatrix}
    1 & e^{i \omega t} & \cdots & e^{i N_H \omega t}
  \end{bmatrix} \in \C^{1 \times (N_H + 1)}
\end{align}
we have
\begin{align}
  \x(t, \omega)
  &= 2 \Re \bbrack*{(\wh Q \otimes I_n) \wh \z} \\
  \f\ts{nl}(\x, \dot \x, \omega)
  &= 2 \Re \bbrack*{(\wh Q \otimes I_n) \wh \bb\ts{nl}} \\
  \f\ts{ext}(t, \omega)
  &= 2 \Re \bbrack*{(\wh Q \otimes I_n) \wh \bb\ts{ext}} \\
  \f(t, \x, \dot \x, \omega)
  &= 2 \Re \bbrack*{(\wh Q \otimes I_n) \wh \bb}.
\end{align}
We also have
\begin{align}
  \label{eq:xp(z)-real}
  \dot \x(t, \omega) &= 2 \omega \Re \bbrack*{\paren{\wh Q \nabla \otimes I_n} \wh \z} \\
  \label{eq:xpp(hat-z)-real}
  \ddot \x(t, \omega) &= 2 \omega^2 \Re \bbrack*{\paren*{\wh Q \nabla^2 \otimes I_n} \wh \z}.
\end{align}
We arrive at the system
\begin{align}
  \label{eq:system-time-kron-real}
  2 \Re
  \bbrack*{
    \paren*{
      \omega^2 \paren*{\wh Q(t) \wh \nabla^2} \otimes M
      + \omega \paren*{\wh Q(t) \wh \nabla} \otimes C
      + \wh Q(t) \otimes K
    } \wh \z
    - \paren*{\wh Q(t) \otimes I_n} \wh \bb
  }
  = 0.
\end{align}
We define
\begin{align}
  \widetilde {\wh \nabla}
  &\coloneq
  \begin{bmatrix}
    0 & i & \cdots & i N_H
  \end{bmatrix} \\
  \label{eq:nabla-real}
  \wh \nabla &\coloneq \diag{\widetilde {\wh \nabla}}.
\end{align}
We then perform Fourier--Galerkin projection on the system.
We project the first term after rewriting the real operator as a sum of the term and its complex conjugate:
\begin{align*}
  &\phantom{{}={}} \frac \omega{2 \pi} \int_0^{2 \pi / \omega} \wh Q(t)^*
  2 \Re \big[
    \omega^2 \paren*{\wh Q(t) \nabla^2} \otimes M
  \big] \, dt \\
  &= \frac \omega{2 \pi} \int_0^{2 \pi / \omega} Q(t)^* \bbrack*{
    \omega^2 \paren*{\wh Q(t) \nabla^2} \otimes M
    + \overline{\omega^2 \paren*{\wh Q(t) \nabla^2} \otimes M}
  } \, dt \\
  &= \omega^2 \frac \omega{2 \pi} \int_0^{2 \pi / \omega} \paren*{
    \wh Q(t)^* \bbrack*{
      \paren*{\wh Q(t) \nabla^2} \otimes M
    } + \wh Q(t)^* \bbrack*{
      \overline{\paren*{\wh Q(t) \nabla^2}} \otimes M
    }
  } \, dt.
\end{align*}
The first summand works out to $\omega^2 \wh \nabla^2 \otimes M$, the only difference in the derivation from that in \ref{sec:projection} being $\wh Q$ instead of $Q$.
For the second summand, we find
\begin{align*}
  &\phantom{{}={}} \omega^2 \frac \omega{2 \pi} \int_0^{2 \pi / \omega}
  \wh Q(t)^*
  \bbrack*{
    \overline{\paren*{\wh Q(t) \nabla^2}} \otimes M
  } \, dt \\
  &= \omega^2 \frac \omega{2 \pi} \int_0^{2 \pi / \omega} \bbrack*{
    \wh Q(t)^* \otimes 1
  } \bbrack*{
    \overline{\paren*{\wh Q(t) \nabla^2}} \otimes M
  } \, dt \\
  &= \omega^2 \frac \omega{2 \pi} \int_0^{2 \pi / \omega} \bbrack*{
    \wh Q(t)^* \overline{\paren*{\wh Q(t) \nabla^2}}
  } \otimes \bbrack*{
    1 M
  } \, dt \\
  &= \omega^2 \bbrack*{
    \frac \omega{2 \pi}
    \int_0^{2 \pi / \omega}
    \wh Q(t)^* \overline{\wh Q(t)} \, dt \; \overline{\nabla^2}
  } \otimes M \\
  &= \omega^2 \big[ \diag(1, 0, \dots, 0)_{N_H + 1} \; \overline{\nabla^2} \big] \otimes M \\
  &= 0_{N_H + 1}.
\end{align*}
Note that each entry of the integral on the third-to-last line is of the form $\inpr{\exp(i k \omega t),\; \exp(-i \ell \omega t)} = \delta_{k(-\ell)}$, and since $k, \ell \ge 0$, only the (0, 0) entry is 1.
On the other hand $\overline{\nabla^2}_{0, 0} = 0$, so the entire expression is zero.

Moving on to the other terms in \eqref{eq:system-time-kron-real}, the non-complex conjugated terms work out as in \ref{sec:projection}.
The complex conjugate of the second term, $\omega \paren*{\wh Q(t) \wh \nabla} \otimes C$, works out to zero like the first term.
We now look at the last two terms.
\begin{align*}
  &\phantom{{}={}} \frac \omega{2 \pi} \int_0^{2 \pi / \omega}
  \wh Q(t)^*
  \bbrack*{
    \overline{\wh Q(t)} \otimes K
  } \, dt \\
  &= \frac \omega{2 \pi} \int_0^{2 \pi / \omega} \bbrack*{
    \wh Q(t)^* \otimes 1
  } \bbrack*{
    \overline{\wh Q(t)} \otimes K
  } \, dt \\
  &= \frac \omega{2 \pi} \int_0^{2 \pi / \omega} \bbrack*{
    \wh Q(t)^* \overline{\wh Q(t)}
  } \otimes \bbrack*{
    1 K
  } \, dt \\
  &= \bbrack*{
    \frac \omega{2 \pi}
    \int_0^{2 \pi / \omega}
    \wh Q(t)^* \overline{\wh Q(t)} \, dt
  } \otimes K \\
  &= \diag(1, 0, \dots, 0)_{N_H + 1} \otimes K.
\end{align*}
The final term (involving $\wh \bb$) becomes $\diag(1, 0, \dots, 0)_{N_H + 1} \otimes I_n$.

Summing the results from the non-complex conjugated terms and the conjugated terms, our algebraic system is
\begin{align*}
  \paren*{
    \omega^2 \wh \nabla^2 \otimes M
    + \omega \wh \nabla \otimes C
    + \diag(2, 1, \dots, 1)_{N_H + 1} \otimes K
  } \wh \z
  - (\diag(2, 1, \dots, 1)_{N_H + 1} \otimes I_n) \wh \bb
  = 0.
\end{align*}
However, we can simplify this in a fortunate (and perhaps surprising) way.
\begin{align*}
  &\phantom{{}={}}
  \paren*{\diag(2, 1, \dots, 1)_{N_H + 1} \otimes I_n} \wh \bb \\
  &= \paren*{\diag(2, 1, \dots, 1)_{N_H + 1} \otimes I_n} H \bb \\
  &= \paren*{\diag(2, 1, \dots, 1)_{N_H + 1} \otimes I_n}
  \paren*{\diag(1/2, 1, \ldots, 1)_{N_H + 1} \otimes I_n} \bb \\
  &= \bb
\end{align*}
\begin{align*}
  &\phantom{{}={}} \paren*{
    \omega^2 \wh \nabla^2 \otimes M
    + \omega \wh \nabla \otimes C
    + \diag(2, 1, \dots, 1)_{N_H + 1} \otimes K
  } \wh \z \\
  &= \paren*{
    \omega^2 \wh \nabla^2 \otimes M
    + \omega \wh \nabla \otimes C
    + \diag(2, 1, \dots, 1)_{N_H + 1} \otimes K
  } H \z \\
  &=
  \paren*{
    \omega^2 \wh \nabla^2 \otimes M
    + \omega \wh \nabla \otimes C
    + \diag(2, 1, \dots, 1)_{N_H + 1} \otimes K
  }
  \paren*{
    \diag(1/2, 1, \ldots, 1)_{N_H + 1} \otimes I_n
  } \z \\
  &= \bbrack*{
    \omega^2 \paren*{\diag(1/2, 1, \ldots, 1)_{N_H + 1} \wh \nabla^2} \otimes M
    + \omega \paren*{\diag(1/2, 1, \ldots, 1)_{N_H + 1} \wh \nabla} \otimes C
    + I_{N_H + 1} \otimes K
  } \z \\
  &= \bbrack*{
    \omega^2 \wh \nabla^2 \otimes M
    + \omega \wh \nabla \otimes C
    + I_{N_H + 1} \otimes K
  } \z.
\end{align*}

We define $\wh A(\omega) \coloneq \paren*{
  \omega^2 \wh \nabla^2 \otimes M
  + \omega \wh \nabla \otimes C
  + I_{N_H + 1} \otimes K
}$
and obtain the system
\begin{align}
  \label{eq:system-real}
  \wh A(\omega) \z = \bb.
\end{align}
Note the wonderful similarity to \eqref{eq:system-time}.

\subsection{Computing the nonlinear force}

We reuse
\begin{align}
  \q(k)
  &\coloneq
  \begin{bmatrix}
    e^{i k \omega t_0} \\
    e^{i k \omega t_1} \\
    \vdots \\
    e^{i k \omega t_{N-1}}
  \end{bmatrix}
\end{align}
and define
\begin{align}
  \wh \Gamma
  &\coloneq
  \begin{bmatrix}
    I_n \otimes \q(0) & I_n \otimes \q(1) & \cdots & I_n \q(N_H)
  \end{bmatrix}.
\end{align}
Then
\begin{align*}
  2 \Re \bbrack*{\wh \Gamma \wh \z}
  &= 2 \Re \bbrack*{
    (I_n \otimes \q(0))
    \wh \ab_{0}
    + \cdots
    + (I_n \otimes \q(N_H))
    \wh \ab_{N_H}
  } \\
  &= \x_s.
\end{align*}
Similarly, defining
\begin{align}
  \wh \Gamma^\dagger &\coloneq \frac1N
  \begin{bmatrix}
    I_n \otimes \q(0)^* \\
    \vdots \\
    I_n \otimes \q(N_H)^*
  \end{bmatrix}
\end{align}
leads to
\begin{align*}
  \wh \Gamma^\dagger \x_s &= \z \\
  H \wh \Gamma^\dagger \x_s &= \wh \z.
\end{align*}
We then can write
\begin{align}
  \label{eq:x_s-real}
  \x_s &= 2 \Re \bbrack*{\wh \Gamma \wh \z} \\
  \label{eq:xp_s-real}
  \dot \x_s &= 2 \omega \Re \bbrack*{\wh \Gamma (\wh \nabla \otimes I_n) \wh \z} \\
  \ddot \x_s &= 2 \omega^2 \Re \bbrack*{\wh \Gamma (\wh \nabla^2 \otimes I_n) \wh \z}.
\end{align}
Additionally, we have
\begin{align}
  \label{eq:b_nl-aft-real}
  \bb\ts{nl}(\z)
  &= \Gamma^\dagger \; \f\ts{nl} \big(
    \x_s,\; \dot \x_s
  \big).
\end{align}

\subsection{Solving the equation}

Define the residual
\begin{align}
  \wh \Rb(\wh \z) \coloneq \wh A(\omega) \z - \bb.
\end{align}
We need to compute
\begin{align*}
  \wh \Rb_{\z}(\z) = \wh A(\omega) + \frac {d \bb\ts{nl}(\z)}{d \z}.
\end{align*}
Observe,
\begin{align*}
  \frac {d \bb\ts{nl}(\z)}{d \z}
  &= \frac d{d \z} \bigg[
    \Gamma^\dagger \f\ts{nl} \paren{\x_s,\; \dot \x_s}
  \bigg] \\
  &= \Gamma^\dagger \frac d{d \z} \bigg[
    \f\ts{nl} \paren{\x_s,\; \dot \x_s}
  \bigg] \\
  &= \Gamma^\dagger \bigg[
    \frac d{d \x} \f\ts{nl} \paren{\x_s,\; \dot \x_s} \frac {d \x_s}{d \z}
    + \frac d{d \dot \x} \f\ts{nl} \paren{\x_s,\; \dot \x_s} \frac {d \dot \x_s}{d \z}
  \bigg] \\
  &= \Gamma^\dagger \bigg[
    \frac d{d \x} \f\ts{nl} \paren{\x_s,\; \dot \x_s} \wh \Gamma
    + \omega \frac d{d \dot \x} \f\ts{nl} \paren{\x_s,\; \dot \x_s} \wh \Gamma (\wh \nabla \otimes I_n)
  \bigg].
\end{align*}

\end{document}