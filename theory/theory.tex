\documentclass{article}

\newcommand{\commandprependpath}{style/}
\usepackage{style/style}

\usepackage[margin=1in]{geometry}

\begin{document}

\section{Introduction}

We implement the method of harmonic balance using exponential periodic basis functions instead of trigonometric.

Assume the state $\x(t): \R \rightarrow \R^n$ is governed by
\begin{align}
  M \ddot \x + C \dot \x + K \x + \f\ts{nl}(\x, \dot \x) = \f\ts{ext}(\omega, t).
  \label{eq:system-time}
\end{align}
Note that $n$ is the number of degrees of freedom;
$M$, $C$, and $K$ are the mass, damping, and stiffness matrices, respectively;
$\f\ts{nl}$ is the internal, potentially nonlinear force;
$\f\ts{ext}$ is the external force;
and $\omega$ is the fundamental frequency of the external force.

Then $x$ is periodic with period $2 \pi / \omega$, so we can estimate it as a finite sum of exponential periodic basis functions:
\begin{align}
  \x(t) \approx \sum_{k=-N_H}^{N_H} \ab_k \exp\paren{i k \omega t}.
\end{align}
We can write the forces similarly:
\begin{align}
  \f\ts{nl}(\x, \dot \x)
  &\approx \sum_{k=-N_H}^{N_H} \cb_k \exp\paren{i k \omega t} \\
  \f\ts{ext}(t)
  &\approx \sum_{k=-N_H}^{N_H} \db_k \exp\paren{i k \omega t}.
\end{align}

\section{Projecting the governing equation}

Collect the Fourier coefficients:
\begin{align}
  \z &\coloneq
  \begin{bmatrix}
    \ab_{-N_H}\trp & \cdots & \ab_{-1}\trp &
    \ab_0\trp & \ab_1\trp & \cdots & \ab_{N_H}\trp
  \end{bmatrix}\trp \in \C^{\paren{2 N_H + 1} n \times 1} \\
  \bb\ts{nl}(\z) &\coloneq
  \begin{bmatrix}
    \cb_{-N_H}\trp & \cdots & \cb_{-1}\trp &
    \cb_0\trp & \cb_1\trp & \cdots & \cb_{N_H}\trp
  \end{bmatrix}\trp \\
  \bb\ts{ext} &\coloneq
  \begin{bmatrix}
    \db_{-N_H}\trp & \cdots & \db_{-1}\trp &
    \db_0\trp & \db_1\trp & \cdots & \db_{N_H}\trp
  \end{bmatrix}\trp.
\end{align}
Also let $\bb(\z) = \bb\ts{ext} - \bb\ts{nl}(\z)$ denote the Fourier coefficients of the force $\f(t, \x, \dot \x) = \f\ts{ext}(t) - \f\ts{nl}(\x, \dot \x)$.

Collect the exponential periodic basis functions
\begin{align}
  Q(t) \coloneq
  \begin{bmatrix}
    e^{-i N_H \omega t} & \cdots & e^{-i \omega t} & 1 &
    e^{i \omega t} & \cdots & e^{i N_H \omega t}
  \end{bmatrix} \in \C^{1 \times (2 N_H + 1)}
\end{align}
so that
\begin{align}
  \x(t)
  &= (Q(t) \otimes I_n) \z \\
  \f\ts{nl}(\x, \dot \x)
  &= (Q(t) \otimes I_n) \bb\ts{nl} \\
  \f\ts{ext}(t)
  &= (Q(t) \otimes I_n) \bb\ts{ext} \\
  \f(t, \x, \dot \x)
  &= (Q(t) \otimes I_n) \bb(\z)
\end{align}
where $\otimes$ denotes the Kronecker product.
Expanding the Kronecker product to make sure we're not pulling our own leg,
\begin{gather*}
  (Q(t) \otimes I_n) \\
  =
  \begin{bmatrix}
    e^{-i N_H \omega t} I_n & \cdots & e^{-i \omega t} I_n & I_n &
    e^{i \omega t} I_n & e^{i N_H \omega t} I_n
  \end{bmatrix} \in \C^{\paren{2 N_H + 1} n \times n}
\end{gather*}
so
\begin{gather*}
  (Q(t) \otimes I_n) \z \\
  =
  \begin{bmatrix}
    e^{-i N_H \omega t} I_n & \cdots & e^{-i \omega t} I_n & I_n &
    e^{i \omega t} I_n & e^{i N_H \omega t} I_n
  \end{bmatrix}
  \begin{bmatrix}
    \ab_{-N_H} \\ \vdots \\ \ab_{-1} \\
    \ab_0 \\ \ab_1 \\ \vdots \\ \ab_{N_H}
  \end{bmatrix} \\
  = \x(t).
\end{gather*}

\subsection{Differentiating}

Suppose we want to write the Fourier series for $\dot \x$?
Observe,
\begin{align*}
  \dot \x(t)
  &= \frac d{dt} \x
  = \frac d{dt} (Q(t) \otimes I_n) \z \\
  &= \paren*{\dot Q(t) \otimes I_n} \z.
\end{align*}
Letting
\begin{align}
  \widetilde \nabla
  &\coloneq
  \begin{bmatrix}
    -i N_H & \cdots -i & 0 & i & \cdots & i N_H
  \end{bmatrix} \\
  \label{eq:nabla}
  \nabla &\coloneq \diag{\widetilde \nabla}
\end{align}
we find that
\begin{align*}
  \dot Q(t)
  &=
  \omega
  \begin{bmatrix}
    -i N_H e^{-i N_H \omega t} & \cdots & -i e^{-i \omega t} & 0 &
    i e^{i \omega t} & \cdots & i N_H e^{i N_H \omega t}
  \end{bmatrix} \\
  &= \omega Q(t) \odot \widetilde \nabla \\
  &= \omega Q(t) \nabla.
\end{align*}
Likewise,
\begin{align*}
  \ddot Q(t)
  &= \omega^2 Q(t) \nabla^2.
\end{align*}
This yields the identities
\begin{align}
  \label{eq:x(z)}
  \x(t) &= \paren{Q(t) \otimes I_n} \z \\
  \label{eq:xp(z)}
  \dot \x(t) &= \omega \paren{Q(t) \nabla \otimes I_n} \z \\
  \label{eq:xpp(z)}
  \ddot \x(t) &= \omega^2 \paren*{Q(t) \nabla^2 \otimes I_n} \z.
\end{align}

\subsection{Performing the projection}

We project the system equation from the time domain \eqref{eq:system-time} to the frequency domain via Fourier--Galerkin projection (I think).
First, define the inner product on functions $f: \R \rightarrow \C$:
\begin{align}
  \inpr{f, g} \coloneq \frac \omega{2 \pi} \int_0^{2 \pi / \omega} \conj{f(t)} g(t) dt.
\end{align}
Now let $-N_H \le k, \ell \le N_H$, and write $\hat k \coloneq k + N_H$ and $\hat \ell \coloneq \ell + N_H$.
Then
\begin{align*}
  &\phantom{{}={}} \frac \omega{2 \pi} \paren*{\int_0^{2 \pi / \omega} Q(t)^* Q(t) dt}_{\hat k \hat \ell} \\
  &= \frac \omega{2 \pi} \int_0^{2 \pi / \omega} \paren*{Q(t)^* Q(t)}_{\hat k \hat \ell} dt \\
  &= \frac \omega{2 \pi} \int_0^{2 \pi / \omega} \exp\paren{-i k \omega t} \exp\paren{i \ell \omega t} dt \\
  &= \inpr{\exp\paren{-i k \omega t},\; \exp\paren{i \ell \omega t}} \\
  &= \delta_{k \ell}
\end{align*}
since the basis functions are orthonormal in this inner product.
Then
\begin{align*}
  \frac \omega{2 \pi} \paren*{\int_0^{2 \pi / \omega} Q(t)^* Q(t) dt}
  = I_{2 N_H + 1}.
\end{align*}

Next, we'll rewrite \eqref{eq:system-time} using \eqref{eq:x(z)}, \eqref{eq:xp(z)}, and \eqref{eq:xpp(z)}, beginning with the first and using the mixed-product property of the Kronecker product:
\begin{align}
  \nonumber
  &\phantom{{}={}} M \ddot \x \\
  \nonumber
  &= M \omega^2 \paren*{Q(t) \nabla^2 \otimes I_n} \z \\
  \nonumber
  &= \omega^2 (1 \otimes M) \paren*{Q(t) \nabla^2 \otimes I_n} \z \\
  \nonumber
  &= \omega^2 \bbrack*{\paren*{1 Q(t) \nabla^2} \otimes (M I_n)} \z \\
  &= \omega^2 \bbrack*{\paren*{Q(t) \nabla^2} \otimes M} \z.
\end{align}
The same computations show
\begin{align}
  C \dot \x &= \omega \bbrack*{\paren*{Q(t) \nabla} \otimes C} \z \\
  K \x &= \bbrack*{Q(t) \otimes K} \z.
\end{align}
Now the system equation \eqref{eq:system-time} becomes
\begin{align}
  \label{eq:system-time-kron}
  \bbrack*{
    \omega^2 \paren*{Q(t) \nabla^2} \otimes M
    + \omega \paren*{Q(t) \nabla} \otimes C
    + Q(t) \otimes K
  }\z
  = (Q(t) \otimes I_n) \bb(\z).
\end{align}
Now we integrate against $Q(t)^*$ to obtain an algebraic system of equations in the frequency domain:
\begin{align*}
  &\phantom{{}={}} \frac \omega{2 \pi} \int_0^{2 \pi / \omega} Q(t)^* \omega^2 \paren*{Q(t) \nabla^2} \otimes M \, dt \\
  &= \frac \omega{2 \pi} \int_0^{2 \pi / \omega} Q(t)^* Q(t) \, dt \; \omega^2 \nabla^2 \otimes M \\
  &= I_{2 N_H + 1} \; \omega^2 \nabla^2 \otimes M \\
  &= \omega^2 \nabla^2 \otimes M.
\end{align*}
Following the same process for the other terms in \eqref{eq:system-time-kron}, we at last obtain
\begin{align}
  \bbrack*{
    \omega^2 \nabla^2 \otimes M
    + \omega \nabla \otimes C
    + I_{2 N_H + 1} \otimes K
  } \z
  = \bb(\z)
\end{align}
and define $A(\omega) \coloneq \bbrack*{
  \omega^2 \nabla^2 \otimes M
  + \omega \nabla \otimes C
  + I_{2 N_H + 1} \otimes K
}$.

\section{Computing the nonlinear force}

How do we compute the Fourier coefficients of the nonlinear force, $\bb(\z)$?
One method is the alternating frequency/time method (AFT), in which take the frequency domain solution $\z$, send it to the time domain to obtain $\x$ and $\dot \x$, pass these through the nonlinear force to obtain $\f\ts{nl}(\x, \dot \x)$, and finally send this back to the frequency domain to obtain $\bb\ts{nl}$.

We also use a fancy technique called trigonometric collocation in which we construct discrete Fourier transform (DFT) matrices using extra time samples DFT so that we can capture information from higher frequencies resulting from the nonlinearity.
(For example, consider passing $e^{i \omega t}$ through a cubic nonlinearity $y \mapsto y^3$, yielding $e^{i 3 \omega t}$.)

\subsection{Computing time from frequency}

Assume we use $N$ samples evenly spaced in one period, $t_j = \frac{2 \pi}\omega \frac jN$, $j = 0, 1, \ldots, N - 1$.
\begin{align}
  \q(k)
  &\coloneq
  \begin{bmatrix}
    e^{i k \omega t_0} \\
    e^{i k \omega t_1} \\
    \vdots \\
    e^{i k \omega t_{N-1}}
  \end{bmatrix} \\
  \Gamma
  &\coloneq
  \begin{bmatrix}
    I_n \otimes \q(-N_H) & \cdots & I_n \otimes \q(0) & I_n \otimes \q(1) & \cdots & I_n \q(N_H)
  \end{bmatrix}
\end{align}
Now
\begin{align*}
  \Gamma \z
  &= (I_n \otimes \q(-N_H)) \ab_{-N_H} + \cdots + (I_n \otimes \q(N_H)) \ab_{N_H}.
\end{align*}
Examining a single summand, we find
\begin{align*}
  [I_n \otimes \q(k)] \ab_k
  &=
  \begin{bmatrix}
    \q(k) & & \\
    & \ddots & \\
    & & \q(k)
  \end{bmatrix}
  \begin{bmatrix}
    a_{k, 0} \\
    \vdots \\
    a_{k, n-1}
  \end{bmatrix} \\
  &=
  \begin{bmatrix}
    \q(k) a_{k, 0} \\
    \vdots \\
    \q(k) a_{k, n-1}
  \end{bmatrix}
\end{align*}
where $a_{k, d}$ is the $k$th Fourier coefficient of the $d$th degree of freedom.
Then
\begin{align*}
  \Gamma \z =
  \begin{bmatrix}
    \sum_{k=-N_H}^{N_H} \q(k) a_{k, 0} \\
    \vdots \\
    \sum_{k=-N_H}^{N_H} \q(k) a_{k, n-1}
  \end{bmatrix}
  =
  \begin{bmatrix}
    \x_0 \\
    \vdots \\
    \x_{n-1}
  \end{bmatrix}
\end{align*}
where $\x_d =
\begin{bmatrix}
  x_d(t_0) & x_d(t_1) & \cdots & x_d(t_{N-1})
\end{bmatrix}
$
contains the values of the $d$th degree of freedom of $\x(t)$ at $t = t_0, \ldots, t_{N-1}$.
Write
\begin{align*}
  \x_s \coloneq \Gamma \z
\end{align*}
to denote this sampling of $\x$ at $N$ points in time.

\subsection{Computing frequency from time}

We now guess at the form of $\Gamma^\dagger$, the pseudo-inverse of $\Gamma$, and check yields the identity when right multiplied by $\Gamma$ (assuming $N \ge 2 N_H + 1$).
We try
\begin{align}
  \Gamma^\dagger \coloneq \frac1N
  \begin{bmatrix}
    I_n \otimes \q(-N_H)^* \\
    \vdots \\
    I_n \otimes \q(N_H)^*
  \end{bmatrix}.
\end{align}
Again using the mixed-product property of the Kronecker product, we note
\begin{align*}
  &\phantom{{}={}} \bbrack*{I_n \otimes \q(k)^*} \bbrack*{I_n \otimes \q(\ell)} \\
  &= I_n \otimes \bbrack*{\q(k)^* \q(\ell)}.
\end{align*}
Next observe that
\begin{align*}
  \frac1N \q(k)^* \q(\ell)
  &= \frac1N \sum_{j=0}^{N-1} e^{-i k \omega t_j} e^{i \ell \omega t_j} \\
  &\approx \frac \omega{2 \pi} \int_0^{2 \pi / \omega}
  e^{-i k \omega t} e^{i \ell \omega t} \, dt \\
  &= \inpr{e^{-i k \omega t},\; e^{i \ell \omega t}} \\
  &= \delta_{k \ell}.
\end{align*}
Then
\begin{align*}
  &\phantom{{}={}} \Gamma^\dagger \Gamma \\
  &= \frac1N
  \begin{bmatrix}
    I_n \otimes \q(-N_H)\trp \\
    \vdots \\
    I_n \otimes \q(N_H)\trp
  \end{bmatrix}
  \begin{bmatrix}
    I_n \otimes \q(-N_H) & \cdots & I_n \q(N_H)
  \end{bmatrix} \\
  &=
  \begin{bmatrix}
    I_n \otimes \bbrack*{\q(-N_H)^* \q(-N_H)} &
    I_n \otimes \bbrack*{\q(-N_H)^* \q(-N_H + 1)} &
    \cdots &
    I_n \otimes \bbrack*{\q(-N_H)^* \q(N_H)} \\
    I_n \otimes \bbrack*{\q(-N_H + 1)^* \q(-N_H)} &
    I_n \otimes \bbrack*{\q(-N_H + 1)^* \q(-N_H + 1)} &
    \cdots & \\
    \vdots & & \ddots & \\
    I_n \otimes \bbrack*{\q(N_H)^* \q(-N_H)} &
    \cdots &
    \cdots &
    I_n \otimes \bbrack*{\q(N_H)^* \q(N_H)} \\
  \end{bmatrix} \\
  &\approx
  \begin{bmatrix}
    \delta_{-N_H, -N_H} & \delta_{-N_H, -N_H + 1} & \cdots & \delta_{-N_H, N_H} \\
    \delta_{-N_H + 1, -N_H} & \delta_{-N_H + 1, -N_H + 1} & \cdots & \delta_{-N_H + 1, N_H} \\
    \vdots & & \ddots & \\
    \delta_{N_H, -N_H} & \delta_{N_H, -N_H + 1} & \cdots & \delta_{N_H, N_H}
  \end{bmatrix} \\
  &= I_{2 N_H + 1}.
\end{align*}
Thus we can compute
\begin{equation}
  \z = \Gamma^\dagger \x_s.
\end{equation}

\subsection{Application}

Recall that given a function $y(t)$ with Fourier coefficients $\bbrace*{\widehat y_k}_{k=-N_H}^{N_H}$, the Fourier coefficients of $\dot y(t)$ are $\bbrace*{i k \omega \widehat y_k}_{k=-N_H}^{N_H}$.
If we collect the Fourier coefficients of $y$ into a vector $\widehat \y =
\begin{bmatrix}
  \widehat y_{-N_H} & \cdots & \widehat y_{N_H}
\end{bmatrix}\trp,$
then using the definition of $\nabla$ \eqref{eq:nabla} we can write the Fourier coefficients of $\dot y$ as
\begin{align*}
  \widehat {\dot y}
  &=
  \begin{bmatrix}
    i (-N_H) \omega \widehat y_{-N_H} & \cdots & i N_H \omega \widehat y_{N_H}
  \end{bmatrix}\trp \\
  &=
  \omega \nabla
  \begin{bmatrix}
    \widehat y_{-N_H} & \cdots & \widehat y_{N_H}
  \end{bmatrix}\trp
\end{align*}

With the Fourier coefficients $\z =
\begin{bmatrix}
  \ab_{-N_H} & \cdots & \ab_{N_H}
\end{bmatrix}$
for the multiple-degree of freedom system $\x(t) \in \R^n$, we can write the Fourier coefficients for $\dot \x(t)$ as
\begin{align*}
  (\omega \nabla \otimes I_n) \z
  &= \omega
  \begin{bmatrix}
    i (-N_H) I_n & \\
    & i (-N_H + 1) I_n & \\
    & & \ddots & \\
    & & & i N_H I_n
  \end{bmatrix}
  \begin{bmatrix}
    \ab_{-N_H} \\
    \ab_{-N_H + 1} \\
    \vdots \\
    \ab_{N_H}
  \end{bmatrix} \\
  &= \omega
  \begin{bmatrix}
    i (-N_H) \ab_{-N_H} & \\
    & i (-N_H + 1) \ab_{-N_H + 1} & \\
    & & \ddots & \\
    & & & i N_H \ab_{N_H}
  \end{bmatrix}.
\end{align*}
Now we can write
\begin{align}
  \label{eq:x_s}
  \x_s &= \Gamma \z \\
  \label{eq:xp_s}
  \dot \x_s &= \omega \Gamma (\nabla \otimes I_n) \z \\
  \ddot \x_s &= \omega^2 \Gamma (\nabla^2 \otimes I_n) \z.
\end{align}
Compare with \eqref{eq:x(z)}, \eqref{eq:xp(z)}, \eqref{eq:xpp(z)}.

Importantly, we can now compute the Fourier coefficients of the nonlinear force $\f\ts{nl}(\x, \dot \x)$:
\begin{align}
  \label{eq:b_nl-aft}
  \bb\ts{nl}(\z)
  = \Gamma^\dagger \f\ts{nl} \big(
    \x_s,\; \dot \x_s
  \big)
  = \Gamma^\dagger \f\ts{nl} \big(
    \Gamma \z,\; \omega \Gamma (\nabla \otimes I_n) \z
  \big).
\end{align}

\section{Solving the equation}

Define the residual equation
\begin{align}
  \Rb(\z)
  &\coloneq \bbrack*{
    \omega^2 \nabla^2 \otimes M
    + \omega \nabla \otimes C
    + I_{2 N_H + 1} \otimes K
  } \z
  + \bb\ts{nl}(\z)
  - \bb\ts{ext} \\
  &= A(\omega) \z
  + \bb\ts{nl}(\z)
  - \bb\ts{ext}
\end{align}
measuring how well a solution $\z$ satisfies the dynamics.
If we have $\Rb_\z(\z) \coloneq d \Rb(\z) / d \z$, we can find roots of this equation using the Newton--Raphson method (and related algorithms).

We compute
\begin{align*}
  \Rb_\z(\z) = A(\omega) + \frac {d \bb\ts{nl}(\z)}{d \z}.
\end{align*}
Then using \eqref{eq:x_s}, \eqref{eq:xp_s}, and \eqref{eq:b_nl-aft},
\begin{align*}
  \frac {d \bb\ts{nl}(\z)}{d \z}
  &= \frac d{d \z} \bigg[
    \Gamma^\dagger \f\ts{nl} \paren{\x_s,\; \dot \x_s}
  \bigg] \\
  &= \Gamma^\dagger \frac d{d \z} \bigg[
    \f\ts{nl} \paren{\x_s,\; \dot \x_s}
  \bigg] \\
  &= \Gamma^\dagger \bigg[
    \frac d{d \x} \f\ts{nl} \paren{\x_s,\; \dot \x_s} \frac {d \x}{d \z}
    + \frac d{d \dot \x} \f\ts{nl} \paren{\x_s,\; \dot \x_s} \frac {d \dot \x}{d \z}
  \bigg] \\
  &= \Gamma^\dagger \bigg[
    \frac d{d \x} \f\ts{nl} \paren{\x_s,\; \dot \x_s} \Gamma
    + \omega \frac d{d \dot \x} \f\ts{nl} \paren{\x_s,\; \dot \x_s} \Gamma (\nabla \otimes I_n)
  \bigg].
\end{align*}

\section{It's getting real}

Since $\x$ is real, we know $\ab_{-k} = \conj{\ab_k}$ and $\exp\paren{-i k \omega t} = \conj{\exp\paren{i k \omega t}}$, so we can write
\begin{align*}
  \x(t)
  &\approx \sum_{k=-N_H}^{N_H} \ab_k \exp\paren{i k \omega t} \\
  &= \ab_0 + \sum_{k=0}^{N_H}
  \ab_k \exp\paren{i k \omega t}
  + \ab_{-k} \exp\paren{-i k \omega t} \\
  &= \ab_0 + \sum_{k=0}^{N_H}
  \ab_k \exp\paren{i k \omega t}
  + \conj{\ab_k \exp\paren{i k \omega t}} \\
  &= \ab_0 + 2 \sum_{k=0}^{N_H} \Re\bbrack{\ab_k \exp\paren{i k \omega t}}.
\end{align*}
We can write the forces similarly:
\begin{align*}
  \f\ts{nl}(\x, \dot \x)
  &= \cb_0 + 2 \sum_{k=0}^{N_H} \Re\bbrack{\cb_k \exp\paren{i k \omega t}}\\
  \f\ts{ext}(t)
  &= \db_0 + 2 \sum_{k=0}^{N_H} \Re\bbrack{\db_k \exp\paren{i k \omega t}}.
\end{align*}
Collect the Fourier coefficients:
\begin{align*}
  \widehat \z &\coloneq
  \begin{bmatrix}
    \ab_0\trp & \ab_1\trp & \cdots & \ab_{N_H}\trp
  \end{bmatrix}\trp \in \C^{n\paren{N_H + 1}} \\
  \widehat \bb\ts{nl} &\coloneq
  \begin{bmatrix}
    \cb_0\trp & \cb_1\trp & \cdots & \cb_{N_H}\trp
  \end{bmatrix}\trp \\
  \widehat \bb\ts{ext} &\coloneq
  \begin{bmatrix}
    \db_0\trp & \db_1\trp & \cdots & \db_{N_H}\trp
  \end{bmatrix}\trp.
\end{align*}

\end{document}